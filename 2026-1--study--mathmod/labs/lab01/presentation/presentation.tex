\input{preamble.tex}

\title{Лабораторная работа № 1 \\ Основы работы в Julia и математическое моделирование}
\author{Сущенко Алина Николаевна \\ НПИбд-01-23}

\institute{Российский университет дружбы народов имени Патриса Лумумбы}
\date{2026}

\begin{document}

\frame{\titlepage}

\begin{frame}
\frametitle{Цель работы}
\begin{itemize}
    \item Подготовка рабочего пространства для выполнения программ и приобретение необходимых навыков создания и преобразования программ на Julia.
\end{itemize}
\end{frame}

\begin{frame}[containsverbatim]
\frametitle{Создание репозитория курса}
Ознакомившись с инструкцией, я перешла к созданию собственного репозитория для курса. Я авторизовалась на сайте GitHub под своей учётной записью. Затем я перешла в репозиторий-шаблон. На странице шаблона я нашла и нажала кнопку «Использовать как шаблон», которая перенаправила меня на страницу создания нового репозитория. В открывшейся форме я указала название для моего нового проекта: \texttt{2026-1--study--mathmod}.

В терминале я выполнила команду создания родительского каталога для всех учебных проектов и перешла в него. Затем я клонировала созданный репозиторий:
\begin{minted}{bash}
git clone https://github.com/ALINASUSHCHENKO/2026-1--study--mathmod.git
\end{minted}
    \centering
    \includegraphics[width=0.8\textwidth]{../images/01.png}
    \captionof{figure}{Клонирование репозитория.}
\end{frame}

\begin{frame}[containsverbatim]
\frametitle{Настройка каталога курса}
Я перешла в каталог клонированного репозитория. Для инициализации структуры каталога под предмет «Математическое моделирование» я записала его код в файл \texttt{COURSE} и свою фамилию в файл \texttt{STUDENT}:
\begin{minted}{bash}
echo mathmod > COURSE
echo "Сушенко Алина Николаевна" > STUDENT
\end{minted}
    \centering
    \includegraphics[width=0.8\textwidth]{../images/02.png}
    \captionof{figure}{Запись информации о курсе и студенте.}
\end{frame}

\begin{frame}[containsverbatim]
\frametitle{Подготовка рабочего пространства}
Затем я запустила процесс подготовки рабочего пространства, выполнив \texttt{make prepare}. После подготовки структуры я добавила все новые файлы в индекс Git, создала первый коммит и отправила файлы на удаленный сервер GitHub:
\begin{minted}{bash}
git add .
git commit -am 'feat(main): make course structure'
git push
\end{minted}
    \centering
    \includegraphics[width=0.8\textwidth]{../images/03.png}
    \captionof{figure}{Добавление файлов и коммит.}
\end{frame}

\begin{frame}[containsverbatim]
\frametitle{Использование Git Flow}
Я инициализировала Git Flow в своём проекте, выполнив команду \texttt{git flow init}. При инициализации я оставила предложенные по умолчанию названия для веток:
\begin{minted}{bash}
git flow init
\end{minted}
    \centering
    \includegraphics[width=0.8\textwidth]{../images/04.png}
    \captionof{figure}{Инициализация Git Flow.}
\end{frame}

\begin{frame}[containsverbatim]
\frametitle{Отправка веток и создание релиза}
Для отправки всей истории и всех веток в удаленные репозитории я использовала команду \texttt{git push -u -{}-all}. Затем я создала релиз с версией 1.0.0:
\begin{minted}{bash}
git push -u --all
git flow release start 1.0.0
git flow release finish 1.0.0
\end{minted}
    \centering
    \includegraphics[width=0.8\textwidth]{../images/05.png}
    \captionof{figure}{Отправка всех веток в удаленный репозиторий.}
\end{frame}

\begin{frame}[containsverbatim]
\frametitle{Отправка тегов}
После завершения релиза я отправила все ветки и теги в удаленный репозиторий:
\begin{minted}{bash}
git push --all
git push --tags
\end{minted}
    \centering
    \includegraphics[width=0.8\textwidth]{../images/07.png}
    \captionof{figure}{Отправка всех веток и тегов.}
\end{frame}

\begin{frame}[containsverbatim]
\frametitle{Создание проекта DrWatson}
Я перешла в каталог, предназначенный для лабораторных работ: \texttt{cd labs/lab01}. В REPL Julia я последовательно выполнила команды для установки и инициализации проекта:
\begin{minted}{julia}
using Pkg
Pkg.add("DrWatson")
using DrWatson
initialize_project("project"; authors="ansusenko", git=false)
\end{minted}
    \centering
    \includegraphics[width=0.8\textwidth]{../images/08.png}
    \captionof{figure}{Установка DrWatson.}
\end{frame}

\begin{frame}[containsverbatim]
\frametitle{Инициализация проекта}
    \centering
    \includegraphics[width=0.8\textwidth]{../images/09.png}
    \captionof{figure}{Инициализация проекта.}
\end{frame}

\begin{frame}[containsverbatim]
\frametitle{Добавление необходимых пакетов}
Следуя инструкции, я последовательно добавила все необходимые для работы пакеты:
\begin{minted}{julia}
Pkg.add("DifferentialEquations")
Pkg.add("Plots")
Pkg.add("DataFrames")
Pkg.add("Literate")
Pkg.add("CSV")
Pkg.add("JLD2")
Pkg.add("IJulia")
Pkg.add("BenchmarkTools")
Pkg.add("Quarto")
\end{minted}
\end{frame}

\begin{frame}
\frametitle{Добавление пакетов (часть 1)}
    \centering
    \includegraphics[width=0.7\textwidth]{../images/10.png}
    \captionof{figure}{Добавление пакета DifferentialEquations.}
    
    \centering
    \includegraphics[width=0.7\textwidth]{../images/11.png}
    \captionof{figure}{Добавление пакета Plots.}
\end{frame}

\begin{frame}
\frametitle{Добавление пакетов (часть 2)}
    \centering
    \includegraphics[width=0.7\textwidth]{../images/12.png}
    \captionof{figure}{Добавление пакета DataFrames.}
    
    \centering
    \includegraphics[width=0.7\textwidth]{../images/13.png}
    \captionof{figure}{Добавление пакета Literate.}
\end{frame}

\begin{frame}
\frametitle{Добавление пакетов (часть 3)}
    \centering
    \includegraphics[width=0.7\textwidth]{../images/14.png}
    \captionof{figure}{Добавление пакета CSV.}
    
    \centering
    \includegraphics[width=0.7\textwidth]{../images/15.png}
    \captionof{figure}{Добавление пакета JLD2.}
\end{frame}

\begin{frame}
\frametitle{Добавление пакетов (часть 4)}
    \centering
    \includegraphics[width=0.7\textwidth]{../images/16.png}
    \captionof{figure}{Добавление пакета IJulia.}
    
    \centering
    \includegraphics[width=0.7\textwidth]{../images/17.png}
    \captionof{figure}{Добавление пакета BenchmarkTools.}
\end{frame}

\begin{frame}
\frametitle{Добавление пакетов (часть 5)}
    \centering
    \includegraphics[width=0.7\textwidth]{../images/18.png}
    \captionof{figure}{Добавление пакета Quarto.}
\end{frame}

\begin{frame}[containsverbatim]
\frametitle{Проверка установки пакетов}
Для проверки корректности установки я создала тестовый скрипт \texttt{scripts/test\_setup.jl}. Я выполнила проверку, запустив этот скрипт с активацией окружения проекта:
\begin{minted}{bash}
cd scripts
julia --project=.. test_setup.jl
\end{minted}
    \centering
    \includegraphics[width=0.8\textwidth]{../images/19.png}
    \captionof{figure}{Запуск тестового скрипта.}
\end{frame}

\begin{frame}[containsverbatim]
\frametitle{Реализация модели экспоненциального роста}
Я создала файл скрипта \texttt{scripts/01\_exponential\_growth.jl} и поместила в него код, реализующий решение дифференциального уравнения экспоненциального роста:
\begin{minted}{julia}
using DifferentialEquations, Plots, DataFrames

# Параметры модели
α = 0.3
u0 = [1.0]
tspan = (0.0, 10.0)

# Определение уравнения
function exponential_growth(du, u, p, t)
    du[1] = α * u[1]
end

# Решение
prob = ODEProblem(exponential_growth, u0, tspan)
sol = solve(prob, Tsit5(), saveat=0.1)
\end{minted}
\end{frame}

\begin{frame}
\frametitle{Создание файла скрипта}
    \centering
    \includegraphics[width=0.8\textwidth]{../images/20.png}
    \captionof{figure}{Создание файла скрипта.}
\end{frame}

\begin{frame}[containsverbatim]
\frametitle{Результат работы скрипта}
Для выполнения этого скрипта я использовала команду \texttt{julia --project=.. 01\_exponential\_growth.jl} (находясь в каталоге \texttt{scripts}):
\begin{minted}{text}
Первые 5 строк результатов:
5×2 DataFrame
 Row │ t          u        
     │ Float64    Float64  
─────┼─────────────────────
   1 │ 0.0        1.0
   2 │ 0.1        1.03045
   3 │ 0.2        1.06184
   4 │ 0.3        1.09417
   5 │ 0.4        1.1275

Аналитическое время удвоения: 2.31
\end{minted}
    \centering
    \includegraphics[width=0.8\textwidth]{../images/21.png}
    \captionof{figure}{Результат работы скрипта.}
\end{frame}

\begin{frame}
\frametitle{График экспоненциального роста}
Скрипт сгенерировал график экспоненциального роста:
    \centering
    \includegraphics[width=0.8\textwidth]{../images/22.png}
    \captionof{figure}{График экспоненциального роста.}
\end{frame}

\begin{frame}
\frametitle{Структура решения дифференциального уравнения}
Модель экспоненциального роста описывается дифференциальным уравнением:
\[ \frac{du}{dt} = \alpha u, \quad u(0) = u_0 \]

Аналитическое решение:
\[ u(t) = u_0 e^{\alpha t} \]

Время удвоения вычисляется по формуле:
\[ T_{1/2} = \frac{\ln 2}{\alpha} \]

Для параметров модели \(\alpha = 0.3\), \(u_0 = 1.0\):
\[ T_{1/2} = \frac{\ln 2}{0.3} \approx 2.31 \]
\end{frame}

\begin{frame}
\frametitle{Структура данных результатов}
Результаты моделирования сохраняются в DataFrame со следующими полями:

\begin{center}
\begin{tabular}{|l|l|c|}
\hline
\textbf{Поле} & \textbf{Описание} & \textbf{Тип данных} \\
\hline
t & Время & Float64 \\
u & Значение популяции & Float64 \\
\hline
\end{tabular}
\end{center}
\end{frame}

\begin{frame}[containsverbatim]
\frametitle{Литературная реализация модели}
Я отредактировала файл \texttt{scripts/01\_exponential\_growth.jl}, добавив подробные комментарии в формате Markdown, чтобы превратить его в литературный исходник:
\begin{minted}{julia}
# # Модель экспоненциального роста
#
# Рассмотрим дифференциальное уравнение:
# $$ \frac{du}{dt} = \alpha u, \quad u(0) = u_0 $$

using DifferentialEquations, Plots, DataFrames

# Параметры модели
α = 0.3
u0 = [1.0]
tspan = (0.0, 10.0)
\end{minted}
    \centering
    \includegraphics[width=0.8\textwidth]{../images/23.png}
    \captionof{figure}{Редактирование кода.}
\end{frame}

\begin{frame}[containsverbatim]
\frametitle{Создание скрипта для генерации форматов}
Я создала скрипт для генерации производных форматов \texttt{scripts/tangle.jl}, скопировав его содержимое из методички:
\begin{minted}{julia}
using Literate
Literate.notebook(ARGS[1], ".")
Literate.markdown(ARGS[1], ".")
Literate.script(ARGS[1], ".")
\end{minted}
    \centering
    \includegraphics[width=0.8\textwidth]{../images/24.png}
    \captionof{figure}{Создание файла скрипта tangle.jl.}
\end{frame}

\begin{frame}[containsverbatim]
\frametitle{Генерация форматов}
С помощью этого скрипта я сгенерировала чистый код, Quarto-документ и Jupyter notebook из моего литературного исходника:
\begin{minted}{bash}
julia --project=.. scripts/tangle.jl scripts/01_exponential_growth.jl
\end{minted}
    \centering
    \includegraphics[width=0.8\textwidth]{../images/25.png}
    \captionof{figure}{Результат работы скрипта tangle.jl.}
\end{frame}

\begin{frame}[containsverbatim]
\frametitle{Реализация параметрического исследования}
Я создала новый, более сложный литературный скрипт \texttt{scripts/02\_exponential\_growth.jl}, который позволял проводить исследование модели с различными параметрами.

После этого я выполнила параметрическое исследование, запустив этот скрипт:
\begin{minted}{bash}
julia --project=.. scripts/02_exponential_growth.jl
\end{minted}
\end{frame}

\begin{frame}
\frametitle{Результат параметрического исследования}
Скрипт успешно выполнил сканирование параметров и сохранил все данные:
    \centering
    \includegraphics[width=0.8\textwidth]{../images/27.png}
    \captionof{figure}{Результат работы параметрического скрипта.}
\end{frame}

\begin{frame}
\frametitle{Параметры исследования}
\begin{center}
\begin{tabular}{|l|l|c|}
\hline
\textbf{Параметр} & \textbf{Значение} & \textbf{Описание} \\
\hline
\(\alpha\) & 0.3 & Коэффициент роста \\
\(u_0\) & [1.0] & Начальная популяция \\
saveat & 0.1 & Шаг сохранения \\
solver & Tsit5() & Численный решатель \\
tspan & (0.0, 10.0) & Интервал времени \\
\hline
\end{tabular}
\end{center}

\vspace{0.5cm}
\textbf{Результаты базового эксперимента:}
\begin{center}
\begin{tabular}{|l|c|}
\hline
\textbf{Показатель} & \textbf{Значение} \\
\hline
Финальная популяция & 20.0854851618676 \\
Время удвоения & 2.31 \\
\hline
\end{tabular}
\end{center}
\end{frame}

\begin{frame}[containsverbatim]
\frametitle{Генерация форматов для параметрического исследования}
Затем я снова воспользовалась скриптом \texttt{tangle.jl} для генерации производных форматов для параметрического исследования:
\begin{minted}{bash}
julia --project=.. scripts/tangle.jl scripts/02_exponential_growth.jl
\end{minted}
    \centering
    \includegraphics[width=0.8\textwidth]{../images/28.png}
    \captionof{figure}{Генерация форматов для параметрического исследования.}
\end{frame}
\begin{frame}
\frametitle{Вывод}
В результате выполнения лабораторной работы:
\begin{itemize}
    \item Создано структурированное рабочее пространство для курса с использованием Git и Git Flow
    \item Установлены и настроены необходимые пакеты Julia
    \item Реализована модель экспоненциального роста с численным решением
    \item Получены результаты, соответствующие аналитическому решению
    \item Освоены принципы литературного программирования с использованием Literate.jl
    \item Сгенерированы различные форматы документов (чистый код, Jupyter notebook, Quarto)
    \item Проведено параметрическое исследование модели
    \item Получены практические навыки организации воспроизводимых научных вычислений
\end{itemize}
\end{frame}

\end{document}
